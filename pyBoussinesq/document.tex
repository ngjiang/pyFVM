\documentclass{article}

\usepackage[english]{babel}
\usepackage[utf8]{inputenc}
\usepackage[T1]{fontenc}

\usepackage{amsmath}
\usepackage{graphicx}
\usepackage{caption}
\usepackage{subcaption}

\usepackage{tikz,pgf}
\usepackage{xcolor}
\usetikzlibrary{decorations.pathmorphing,shadows,positioning} 
\usetikzlibrary{patterns}
\usetikzlibrary{arrows,shapes}
%\usepackage{beamerfoils,tikz,pgf}
\usepackage{hyperref}


%\graphicspath{{../figures/}}
\usepackage[a4paper, left=2cm, right=3cm, top=2cm, bottom=3cm]{geometry}
%\newcommand{\bm}[1]{\mbox{\boldmath{$#1$}}}

\usepackage{amsmath}
\usepackage{enumerate}

\title{Solution of Boussinesq equation with Python}
%namespace Blasius
\author{JIANG Nan (Gabriel) \\ under direction of \\ Bérengère PODVIN and Etienne STUDER}
%\date{December 8, 2016\\ \it first modification: December 12, 2016\\ second modification: December 14, 2016}
\begin{document}
\maketitle
%\tableofcontents
%\newpage

\section{Equations}
We solve the non-dimensionalized Boussinesq equations:
\begin{equation}
\nabla \cdot U = 0
\end{equation}
\begin{equation}
\frac{\partial}{\partial t} U + U \cdot \nabla U 
- \frac{1}{\text{Re}} \nabla^2 U
= - \nabla p_d + T \frac{-g}{\|g\|}
\end{equation}
\begin{equation}
\frac{\partial}{\partial t} + U \cdot \nabla T
- \frac{1}{\text{Re} \cdot \text{Pr}} \nabla^2 T = 0
\end{equation}




%\bibliographystyle{../iso690}
%\bibliography{../references}

\end{document}
